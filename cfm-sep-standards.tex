\documentclass{article}
\usepackage{setspace}

\begin{document}

\centerline{\sc \large How can I help others understand my standards?}
\vspace{.5pc}
\centerline{\sc Sunday School course 17--18}
\centerline{\sc 27 September 2015}
\vspace{.5pc}
\centerline{\scriptsize (see https://www.lds.org/youth/learn/ss/commandments/standards)}
\vspace{6pc}

\section*{References}
\begin{itemize}
  \item 2 Nephi 8:7
  \item Romans 1:16
  \item 2 Timothy 1:7--8
  \item 1 Timothy 4:12
  \item 3 Nephi 11:29
  \item D\&C 11:21; 84:85; 100:5--8
  \item Answering Gosepl Questions (https://www.lds.org/topics/answering-gospel-questions)
\end{itemize}

\section*{Story}
Working in the naval program at Oak Ridge, Tennessee, Elder Scott completed the equivalent of a doctorate in nuclear engineering. Because the field was top secret, a degree could not be awarded. The naval officer who invited young Richard Scott to join the nuclear program was Hyman Rickover, a pioneer in the field. They worked together for 12 years -- until Elder Scott was called to serve as mission president in Argentina in 1965. Elder Scott explained how he received the call:

``I was in a meeting one night with those developing an essential part of the nuclear power plant. My secretary came in and said, `There's a man on the phone who says if I tell you his name you'll come to the phone.'

``I said, `What's his name?'

``She said, `Harold B. Lee.'

``I said, `He's right.' I took the phone call. Elder Lee, who later became President of the Church, asked if he could see me that very night. He was in New York City, and I was in Washington, D.C. I flew up to meet him, and we had an interview that led to my call to be a mission president.''

Elder Scott then felt he should immediately let Admiral Rickover, a hardworking and demanding individual, know of his call.

``As I explained the mission call to him and that it would mean I would have to quit my job, he became rather upset. He said some unrepeatable things, broke the paper tray on his desk, and in the comments that followed clearly established two points:

`` `Scott, what you are doing in this defense program is so vital that it will take a year to replace you, so you can't go. Second, if you do go, you are a traitor to your country.'

``I said, `I can train my replacement in the two remaining months, and there won't be any risk to the country.'

``There was more conversation, and he finally said, `I never will talk to you again. I don't want to see you again. You are finished, not only here, but don't ever plan to work in the nuclear field again.' ''

``I responded, `Admiral, you can bar me from the office, but unless you prevent me, I am going to turn this assignment over to another individual.' ''

True to his word, the admiral ceased to speak to Elder Scott. When critical decisions had to be made, he would send a messenger. He assigned an individual to take Elder Scott's position, whom Elder Scott trained.

On his last day in the office, Elder Scott asked for an appointment with the admiral. His secretary was shocked. Elder Scott entered the office with a copy of the Book of Mormon. Elder Scott explained what happened next:

``He looked at me and said, `Sit down, Scott, what do you have? I have tried every way I can to force you to change. What is it you have?' There followed a very interesting, quiet conversation. There was more listening this time.

``He said he would read the Book of Mormon. Then something happened I never thought would occur. He added, `When you come back from the mission, I want you to call me. There will be a job for you.' ''

Elder Scott shared the lesson he learned from this and the many other times he chose the right despite opposition: ``You will have challenges and hard decisions to make throughout your life. But determine now to always do what is right and let the consequence follow. The consequence will always be for your best good.''

\subsection*{Discussion}
\begin{enumerate}
  \item How did Elder Scott prioritize his life?
  \item Describe the pressure under which Elder Scott found himself. How did he deal with a high ranking commanding officer? Do you ever find yourself in similar situations?
  \item What can we learn from Elder Scott's understanding and application of our responsibility to share the gospel (see references).
\end{enumerate}

\section*{Obedience}
From \textit{True to the Faith}, (2004), 108--9:

\begin{quotation}
Many people feel that the commandments are burdensome and that they limit freedom and personal growth. But the Savior taught that true freedom comes only from following Him: ``If ye continue in my word, then are ye my disciples indeed; and ye shall know the truth, and the truth shall make you free'' (John 8:31--32). God gives commandments for your benefit. They are loving instructions for your happiness and your physical and spiritual well-being.
\end{quotation}

\begin{enumerate}
  \item How would you respond to a friend who ways the commandments are too restrictive?
  \item What scriptures, examples, or personal experiences could you share with a friend to help him or her understand the purposes of God's commandments?
\end{enumerate}

\section*{Role-play}
Practice explaining your standards in various situations.

\end{document}
